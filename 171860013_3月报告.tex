\documentclass[a4paper,UTF8]{article}
\usepackage{ctex}
\usepackage[margin=1.25in]{geometry}
\usepackage{color}
\usepackage{graphicx}
\usepackage{amssymb}
\usepackage{amsmath}
\usepackage{amsthm}
%\usepackage[thmmarks, amsmath, thref]{ntheorem}
\theoremstyle{definition}
\newtheorem*{solution}{Solution}
\newtheorem*{prove}{Proof}
\usepackage{multirow}
\usepackage{url}
\usepackage[colorlinks,urlcolor=blue]{hyperref}
\usepackage{enumerate}
\renewcommand\refname{参考文献}


%--

%--
\begin{document}
\title{\textbf{《计算机图形学》3月报告}}
\author{171860013,刘恩萌,\href{171860013@smail.nju.edu.cn}{171860013@smail.nju.edu.cn}}
\maketitle

\section{综述}
\begin{itemize}
  \item 完成内容:本月以熟悉框架为主,实现了cg\_algorithms.py中的大部分算法,以及cg\_cli.py中对应的操作来测试算法的实现
  \item 开发环境:Ubuntu 18.04 Python 3.7.0
  \item 内容为$\dots$的部分表示尚未实现
\end{itemize}


% 已完成或拟采用算法的原理介绍、自己的理解、对比分析等
% 已完成或拟采用的系统框架、交互逻辑、设计思路等
% 介绍自己系统中的巧妙的设计、额外的功能、易用的交互、优雅的代码、好看的界面等(可选)
\section{算法介绍}
\subsection{直线绘制}
按照实验要求实现了DDA与Bresenham直线绘制算法。
\subsubsection{DDA算法}
\begin{itemize}
  \item 原理
  \begin{itemize}
    \item 使用$x$或$y$方向单位增量间隔,逐步计算沿线路径上各像素点的位置
    \item 在增量绝对值更大的方向上按照单位增量对线段离散取样,利用增量的比例计算出取样点在另一个方向上的坐标(此处需要取整操作)
  \end{itemize}
  \item 实现\cite{ddawiki}
  \begin{enumerate}
    \item 对于输入起始点\texttt{(x0,y0), (x1,y1)},先计算出各自的增量\texttt{dx, dy}
    \item 选取增量中绝对值较大的一者为\texttt{step=max(abs(dx), abs(dy))}
    \item $x,y$方向每次的增量分别为\texttt{dx/step, dy/step}
    \item 两个方向上每次递增增量并取整(\texttt{round()}),计算出\texttt{step+1}个点组成直线
  \end{enumerate}
  \item 分析
  \begin{itemize}
    \item 优点:符合数学直观,易于理解和记忆;消除了直线方程 中的乘法
    \item 缺点:每个点的像素位置都需要面临取整,可能会产生累积误差(长线段所计算的像素位置可能会偏离实际线段);涉及大量浮点运算和取整操作,比较耗时
  \end{itemize}
\end{itemize}

%%%%%%%%%%%%%%%%%%%%%%%%%%%%%%%%%%%%%%%%%

\subsubsection{Bresenham算法}
\begin{itemize}
  \item 原理
  \begin{itemize}
    \item 利用直线点的连续性,从起点开始,(斜率$0<m<1$的直线)每次选取的下一个点要么是当前点右侧相邻的点,要么是当前点右上角相邻的点
    \item 分别计算两个候选点与直线方程计算得到的实际点的偏移$d_1$(右侧点的偏移)和$d_2$(右上角点的偏移),总是选择离实际点较近的那个点
    \item 对于偏移量不进行真正的浮点计算,而是使用作差法并将计算公式变形并化简为仅包含整数运算的\textup{决策参数}$p_k=\Delta x(d_1-d_2)=2\Delta yx_k-2\Delta xy_k+c$($c$为常数)。根据第$k$步决策参数的符号,判定两候选像素与线段的偏移关系。
    \item 从第$k$步到第$k+1$步只需要计算决策参数的增量就可以更新决策参数
    $$p_{k+1}=p_{k}+2\Delta y-2\Delta x(y_{k+1}-y_k)=
    \begin{cases}
    p_k+2\Delta y-2\Delta x &p_k>0\;(y_{k+1}=y_k+1)\\
    p_k+2\Delta y & p_k<0\;(y_{k+1}=y_k)
    \end{cases}$$
    决策参数初始值为$p=2\Delta y-\Delta x$。
  \end{itemize}
  \item 实现 \cite{rog_2002}
  \begin{enumerate}
    \item 输入直线的起始点\texttt{(x0,y0), (x1,y1)},如果发现是水平或垂直线(\texttt{x0 == x1 or y0 == y1})直接输出所有端点。
    \item 计算常量:\texttt{dx, dy},由于实现基于Python而非硬件环境,且对效率的要求并不苛刻,每次计算$2\Delta y$和$2\Delta y-2\Delta x$的时间可以忽略不计,故此处不作存储。
    \item 判断直线生成方向是否与坐标轴方向一致(\texttt{(m > 0 and x0 < x1) or (m < 0 and y0 < y1)}),不一致时交换起始点坐标
    \item 按照斜率绝对值和1的关系(\texttt{dx, dy}的相对大小)分情况处理。按照原理部分所述的步骤计算每步的决策参数,决定当前加入集合的点。
  \end{enumerate}
  \item 实现时遇到的困难及解决
  \begin{itemize}
    \item 按照ppt给出算法实现的时候,ppt上只给出了斜率$m>1$时的做法,我也只按照斜率的绝对值与1的大小关系区分了实现。测试时才发现$m<-1$时决策参数的公式是不一样的。正确的做法应该是1)$\Delta x$和$\Delta y$需要取绝对值,2)$p_k>$时应有$y_{k+1}=y_k-1$,因为候选点是当前点右侧或右下角的相邻点。
  \end{itemize}
  \item 分析
  \begin{itemize}
    \item 优点:全都是整数计算,硬件实现非常容易,也不会出现累计误差
    \item (不算缺点的)缺点:计算有些琐碎,有8种不一样的直线情况需要分类讨论(有些情况可以合并),需要一定时间理解和消化
  \end{itemize}
\end{itemize}

\subsection{椭圆绘制}
按照要求,实现的是Bresenham中点椭圆生成算法。
\begin{itemize}
  \item 原理\cite{rog_2002}
  \begin{itemize}
    \item 利用\textbf{平移}和\textbf{对称性},只需要生成第一象限部分的曲线。而第一象限被斜率为-1的切线的切点分为了两部分。前半部分$\Delta x$较大,以$x$轴为基准计算;后半部分$\Delta y$较大,反之。下讨论前半部分的绘制思路。
    \item 与Bresenham直线绘制算法类似,根据椭圆线段的连续性,下一个像素必然是当前点$(x_k, y_k)$的右侧像素$(x_k+1,y_k)$或右下侧像素$(x_k+1, y_k-1)$。考察这两个候选像素的中点$(x_k+1, y_k-1/2)$与椭圆$f(x,y)$的位置关系可以得知哪个点与实际曲线上的点离得更近。
    \begin{table}[h]
      \centering
      \begin{tabular}{c|c|c}
        \hline\hline
        $f(x_k+1, y_k-1/2)<0$ & 中点位于椭圆内 & 右侧像素与实际点更近\\\hline
        $f(x_k+1, y_k-1/2)=0$ & 中点位于椭圆上 & 右侧像素和右下侧像素与实际点距离相同\\\hline
        $f(x_k+1, y_k-1/2)>0$ & 中点位于椭圆外 & 右下侧像素与实际点更近\\\hline\hline
      \end{tabular}
    \end{table}
    \item 经过化简,可以将$p1_k=f(x_k+1,y_k-1/2)=r_y^2(x_k+1)^2+r_x^2(y_k-1/2)-r_xr)y^2$作为决策参数,根据其符号来决定下一个像素的选择。初值为$p1_0=r_y^2-r_x^2*r_y+r_x^2/4$。
    \item 同Bresenham直线绘制算法,可以使用增量来更新决策参数,即
    $$p1_{k+1}=\begin{cases}
      p1_k+2r_y^2x_{k+1}+r_y^2 & p1_k < 0\\
      p1_k+2r_y^2x_{k+1}-2r_x^2y_{k+1}+r_y^2 & p1_k>0
    \end{cases}$$
    \item 第一部分计算到$2r_y^2x\ge 2r_x^2y$为止(斜率为-1的切线与椭圆的交点)。计算椭圆第二部分的原理类似。
    \item 第二部分更换$y$轴为递增单位。决策参数初值为$p2_0=r_y^2(x_1+1/2)^2+r_x^2(y_1-1)^2-r_x^2r_y^2$。候选像素点为下方$(x_k, y_k-1)$和右下方$(x_k+1,y_k-1)$的像素。增量更新公式为
    $$p2_{k+1}=\begin{cases}
      p2_k-2r_x^2y_{k+1}+r_x^2 & p2_k>0\\
      p2_k+2r_y^2x_{k+1}-2r_x^2y_{k+1}+r_x^2 & p2_k<0
    \end{cases}$$
    循环至$(r_x,0)$处。
  \end{itemize}
  \item 实现
  \begin{itemize}
    \item 输入椭圆的外接矩形的左上角和右下角坐标\texttt{(x0,y0), (x1,y1)},计算出椭圆的长轴、短轴和圆心位置。
    \item 计算出几个常用数值\texttt{rx2=rx*rx, ry2=ry*ry}
    \item 按照上述原理,递增计算每一步的决策参数选择像素点,分为两部分绘制椭圆的第一象限中的曲线。
    \item 将第一象限中的部分对称到其他三个象限中,并整体平移到圆心指示的位置上。
  \end{itemize}
  \item 实现中遇到的问题及解决
  \begin{itemize}
    \item 起初按照ppt上给出的过程实现的,会出现“烈焰红唇”一样的形状,一步一步推导后才发现ppt中第二部分的初始值公式中少打了一个平方符号。深刻体会到了自己理解算法的重要性,不然就被坑了呀。
  \end{itemize}
  \item 分析
  \begin{itemize}
    \item 优缺点基本同Bresenham直线绘制算法
  \end{itemize}
\end{itemize}
%%%%%%%%%%%%%%%%%%%%%%%%%%%%%%%%%%%%%%%%%%%%%%%
\subsection{多边形绘制}
\begin{itemize}
  \item 直接使用了框架提供的代码:在输入点两两之间绘制直线相连
\end{itemize}

%%%%%%%%%%%%%%%%%%%%%%%%%%%%%%%%%%%%%%%%%%%%%%%
\subsection{曲线绘制}
\subsubsection{Brezier曲线}
\dots
\subsubsection{三次B样条曲线}
\dots

%%%%%%%%%%%%%%%%%%%%%%%%%%%%%%%%%%%%%%%%%%%%%%%%
\subsection{平移变换}
\begin{itemize}
  \item 原理
  \begin{itemize}
    \item 将待平移图形的每个坐标加上需要平移的量
  \end{itemize}
  \item 实现
  \begin{itemize}
    \item 实际实现中,只需平移每种图形的“定位点”(直线的端点、多边形的所有顶点、椭圆的外接矩形顶点、曲线的定位点等),返回平移后的点并调用绘制算法重新绘制这些曲线即可
  \end{itemize}
\end{itemize}

%%%%%%%%%%%%%%%%%%%%%%%%%%%%%%%%%%%%%%%%%%%%%%%%%
\subsection{旋转变换}
\begin{itemize}
  \item 原理
  \begin{itemize}
    \item 先考虑旋转中心为坐标原点的情况:利用极坐标表示可以得到平移变换后点的计算公式
    $$\begin{cases}
      x_1=x\cos\theta-y\sin\theta\\
      y_1=x\sin\theta+y\cos\theta
    \end{cases}$$
    即变换矩阵为$$R=\begin{bmatrix}
      \cos\theta & -\sin\theta\\
      \sin\theta & \cos\theta
      \end{bmatrix}$$
    \item 当旋转中心非坐标原点时,可以暂时以旋转中心$(x_r,y_r)$为原点建立临时坐标系,再讲旋转后的图形坐标变换回到原坐标系中即可,即
    $$\begin{cases}
      x_1=x_r+(x-x_r)\cos\theta-(y-y_r)\sin\theta\\
      y_1=y_r+(x-x_r)\sin\theta+(y-y_r)\cos\theta
    \end{cases}$$
  \end{itemize}
\end{itemize}

%%%%%%%%%%%%%%%%%%%%%%%%%%%%%%%%%%%%%%%%%%%%%%%%%%
\subsection{缩放变换}
\dots

%%%%%%%%%%%%%%%%%%%%%%%%%%%%%%%%%%%%%%%%%%%%%%%%%%%
\subsection{线段裁剪}
\subsubsection{Cohen-Sutherland算法}
\dots
\subsubsection{Liang-Barsky算法}
\dots
    
%%%%%%%%%%%%%%%%%%%%%%%%%%%%%%%%%%%%%%%%%%%%%%%%%%%%

\section{系统介绍}
\subsection{CLI部分}
\begin{itemize}
  \item 基本采用原框架代码的命令解释器架构:分类讨论各个输入命令,提取命令中的参数,调用相应的绘制算法,或创建保存的动作
  \item 新增对支持注释,输入文件中以\#开头的行全部视为注释不予解释
\end{itemize}

\subsection{GUI部分}
\dots

\section{总结}
实验尚未成功,同志任需努力

\bibliographystyle{plain}%
%"xxx" should be your citing file's name.
\bibliography{cgref}

\end{document}